\documentclass[twoside,12pt]{article} 
\usepackage{amsmath,amssymb,varioref}
\usepackage{xspace,verbatim,listings,fullpage}
\usepackage{color,tikz}
\lstset{basicstyle=\ttfamily\footnotesize,numbers=left,numberstyle=\scriptsize,columns=fullflexible,backgroundcolor=\color{purple!10!white}}
\newcommand{\qpic}{$\langle\mathsf{q}|\mathsf{pic}\rangle$\xspace}
\newcommand{\qpicpy}{{\tt qpic}\xspace}
\newcommand{\TikZ}{Ti\emph{k}Z\xspace}
\providecommand{\ket}[1]{\left|#1\right\rangle}
\usetikzlibrary{decorations.pathreplacing,decorations.pathmorphing}
\usepackage{hyperref}

\title{$\langle\mathsf{q}|\mathsf{pic}\rangle$: Quantum Circuit Diagrams in \LaTeX}
\author{Thomas G. Draper \and Samuel A. Kutin}
\date{February 2016}

\begin{document}
\maketitle
\begin{abstract} 
{$\langle\mathsf{q}|\mathsf{pic}\rangle$\xspace} is a system
for preparing circuit diagrams in \LaTeX, with an emphasis on diagrams
used in quantum computing.  The user prepares a description of the
circuit in the human-readable
``{$\langle\mathsf{q}|\mathsf{pic}\rangle$\xspace} language'':\ a
Python program then converts this into {\LaTeX} code using the
{Ti\emph{k}Z\xspace} graphics package.  This note serves as a manual
for the language as of
{$\langle\mathsf{q}|\mathsf{pic}\rangle$\xspace} version~5.0.1a
(February 2016).
\end{abstract}

\tableofcontents
\newpage

\section{Introduction}

A picture is worth approximately a thousand words.  For example, Figure~\ref{fig-rev} depicts a
circuit~\cite{kutin-moulton-smithline} for reversing the information on $n$ wires in depth $2n+2$,
where the allowed operation is adding one wire to an adjacent wire.  The proof that this circuit
is correct takes a page and a half (and over 600 words), but the main idea of the proof is conveyed
entirely by the red coloring in Figure~\ref{fig-rev}.

\begin{figure}[h!]
\begin{center}
\input reverse-8.tikz
\end{center}
\caption{Reversing the contents of nine wires in depth~20~\cite{kutin-moulton-smithline} using controlled NOTs.  The red wires and gates are those affected by the value of $a_3$.}
\label{fig-rev}
\end{figure}


This diagram was created using \qpic.  The user creates a description of the
circuit in \qpic: a simple, human-readable language, with one line declaring each wire and one line specifying each
logical operation, or ``gate''.  A Python script, which we
call\footnote{\qpicpy stands for ``\qpic Python-interpreted compiler''.}
\qpicpy, parses this description
and produces {\LaTeX} code using the \TikZ graphics package.\footnote{The bulk of our testing has been with \TikZ version 2.10.}

One design principle behind \qpic is to keep the language simple.  The user can draw a number of
elements common to quantum circuit diagrams, including wires with single or double lines, rectangular
gates with text on them, and measurement boxes.  The code is interpreted by \LaTeX, so any math expressions
are passed directly through.  The goal is not to handle everything anyone might ever want; instead,
commands can be passed directly through to {\LaTeX} or to \TikZ, so the user can supplement the \qpic output
as needed.  (\qpicpy produces heavily commented \TikZ code to facilitate this process.)


There are, of course, some limitations on what \qpic can achieve in this framework.  The Python code
that creates the \TikZ commands needs to know how large all the elements are; so, for example, to make
a box large enough to contain some text, the user must explicitly specify the width.  Hence, it may
take several iterations to line up the diagram perfectly, as is normal when constructing diagrams in \LaTeX.


Another design principle is that, to the extent possible, the \qpic interpreter
first creates an internal representation
of the circuit from the user's input, and then produces \TikZ code from that representation.  One example
of this is the {\tt VERTICAL} command; with one line, the user can change the default flow of time in the
diagram (left to right) to a vertical flow (top to bottom).  The internal representation of the circuit is
the same in horizontal or vertical mode; only the output changes.  In practice, the line between
representation and code is not always as clear-cut as one would like.


In this paper we give a brief, but complete, description of \qpic.  We begin with some simple examples in
Section~\ref{sec-example}; the discussion should serve as a beginning tutorial.  
We give a complete list of commands of the \qpic circuit specification language
in Section~\ref{sec-qpic}.  We then briefly describe \qpicpy in Section~\ref{sec-tools}, including
tools to help users use \qpic within a {\LaTeX} document.  Finally, we complete our definition of the
\qpic language by listing the gory details of parsing and tokenizing in Section~\ref{sec-tokens}.

Like any software, \qpic is a work in progress.  This paper represents a description of the state of
the program as of June 2015.  Please see {\tt github} for the most up-to-date version of \qpicpy and \qpic.  \qpic is copyright IDA/CCR-P, and is distributed
under GNU General Public License v3. 
Please feel free to contact the authors with suggestions for further improvement.

\section{Simple Examples}
\label{sec-example}
\subsection{Example 1:\ Majority}
\label{sec-example-maj}
We begin with a simple in-place majority circuit~\cite{CDKM} in Figure~\ref{fig-maj}.  The figure includes the diagram and also the \qpic code used to generate it.
\begin{figure}[h!]
\begin{center}
\input Adder_CDKM_MAJ.tikz

\begin{minipage}{2in}
%\footnotesize\verbatiminput{Adder_CDKM_MAJ.qpic}
\lstinputlisting{Adder_CDKM_MAJ.qpic}
\end{minipage}
\end{center}
\caption{In-place majority vote:\ diagram and code.}
\label{fig-maj}
\end{figure}

The first three lines of the \qpic code use the {\tt W} command, which declares a wire.  The string before {\tt W}, which is required, is the name of the wire, to be used internally.  The formal rules for wire names can be found in
Section~\ref{sec-qpic-wire-name-rules}---in brief, a wire name can be anything that doesn't mean something else to \qpic---but any string of numbers and lowercase
letters is safe.  Here, the wires are given the names {\tt a}, {\tt b}, and
{\tt c}.


If a {\tt W} command has an argument after the {\tt W}, it is interpreted as a label to be
typeset in math mode and placed at the start of the wire.  A second argument (if any) is treated the same way but placed at the end.  Here, each wire has a starting and ending label, representing its starting and ending value.


Normally, the \qpic parser breaks up each line by whitespace.  However,
an expression like {\tt \{a{\textvisiblespace}{\char92}oplus{\textvisiblespace}c\}} is not
broken up.  \qpic's target language is {\LaTeX}, so curly
braces (and dollar signs) can be used to group text with whitespace into
a single entity.  See Section~\ref{sec-tokens} for more about \qpic's parsing
rules.


After a blank line (ignored by \qpic), line 5 contains our first gate.
The basic syntax of a gate line is a list of targets, the gate (possibly
with a required argument), and then a list of controls (if any).  In this case,
the gate type is {\tt G}, which means a rectangle drawn around the targets.
The required argument following the {\tt G} is the name of the gate, which
is placed in the center of the rectangle.  It is processed by \TikZ, so it
may include graphics commands (here, {\tt {\char92}rotatebox}).  Gate names
may be enclosed in dollar signs to be typeset in math mode.


In addition to wires and gates, \qpic allows for several other common
elements of circuit diagrams.  The {\tt =} command in line 6 takes one
optional argument and draws it in the center of the circuit with a white
background.  If, as in this case, no argument is given, an equals sign is
drawn.


The three remaining lines contain additional gates: two controlled NOTs and
one Toffoli.  One can use {\tt C} for controlled NOT and {\tt T} for
Toffoli; for example, line 7 could have been written {\tt b C c}, a
controlled NOT with target $b$ and control $c$.  However, since these gates
are so common, \qpic interprets a line containing only a list of wires as
such a gate:  {\tt +} indicates that the operator $\oplus$ should be
applied to that wire, and any other wires are treated as controls.
Notice that the target may occur anywhere in the list.  If no target is
listed, all wires are drawn as controls, indicating a (controlled)
$Z$ gate.

\subsection{Example 2:\ Quantum Fourier Transform}
\label{sec-example-QFT}

Our second example is a 3-bit QFT, shown in Figure~\ref{fig-QFT}.

%moved figure after another paragraph because of page endings.


The {\tt PREAMBLE} commands insert {\LaTeX} code before the
\TikZ environment; in this case, we define the commands {\tt {\char92}ket}
and {\tt {\char92}phase}, which we will use in the wire labels.
({\tt {\char92}providecommand} is
useful for this construct, since it defines the command only if it is not
already defined.)



The {\tt SCALE} command scales the graphical elements in the picture,
here by a factor of 1.5.  Note that text is not scaled.


In lines 4--6 we declare the wires, as discussed in
Section~\ref{sec-example-maj}.  Note that the internal wire names have
nothing to do with the depiction in the diagram.

The remaining lines contain two types of gates:
{\tt H} (Hadamard) and {\tt P} (phase shift).  {\tt H} is required to have
one target, and {\tt P} typically does as well.
For a Hadamard, the {\tt H} completely specifies the gate;
for a phase shift (as with the rectangle in Section~\ref{sec-example-maj})
we must also specify something to be written inside the circle.  In this
example the phase shifts have controls and the Hadamard gates do not, but
either gate type is allowed to have an arbitrary number of controls.
\begin{figure}[h!]
\begin{center}
\input QFT3v1.tikz

\begin{minipage}{5in}
\lstinputlisting{QFT3v1.qpic}
\end{minipage}
\end{center}
\caption{Quantum Fourier transform on three bits:\ diagram and code.}
\label{fig-QFT}
\end{figure}


It is worth looking at the spacing between gates in Figure~\ref{fig-QFT}.
Most of the gates are separated by horizontal space, but there is less space
between the second phase gate and the Hadamard on $\ket{x_1}$.  This is
because \qpic detects that these two gates operate on disjoint sets of
wires and may be performed in parallel.  (If we changed the order of the
wires, \qpic would draw the Hadamard directly above or below the
phase shift; in Figure~\ref{fig-QFT}, the two gates are simply next to each other.)


\qpic uses a simple greedy algorithm to determine gate sequences:  it
divides the circuit into a series of time slices, or \emph{slices}, placing each gate
in the earliest possible slice.  This is one of the key features of
\qpic; the user can specify the list of gates in logical order without
worrying about exactly how they will appear on the page.  Of course, there
are ways to override this default behavior as needed; see
Section~\ref{sec-qpic-touch} for more information.

\subsection{Example 3:\ Shor's Algorithm}


Figure~\ref{fig-Shor} illustrates some additional features of \qpic, and
also a different application:\ a high-level schematic of an algorithm.


\begin{figure}[h!]
\begin{center}
\includegraphics{ShorNutshell}
\begin{minipage}{5.4in}
\lstinputlisting[basicstyle=\scriptsize\ttfamily,numberstyle=\tiny,breaklines=true]{ShorNutshell.qpic}
\end{minipage}
\end{center}
\caption{Shor's algorithm:\ diagram and code.}
\label{fig-Shor}
\end{figure}


Line 1 contains the {\tt VERTICAL} command; this tells \qpic to render the
circuit vertically, with time flowing downward.  By default the top and
bottom labels on the wires are typeset at a 45-degree angle; the {\tt 0}
indicates that they should instead be written horizontally.  The other
new command at the top, {\tt DEPTHPAD 3}, changes the spacing between
slices to 3 points.


Lines 6 and 7 look like standard wire declarations, except for the
{\tt width=25} on line 7.  This indicates that \qpic needs to leave more space
for this wire on the page, to accomodate the wider exponentiation gate.
As we will discuss in Section~\ref{sec-qpic-style}, we could also have said
{\tt breadth=25}.  Note that {\tt \#} is a comment
character in \qpic; the remainder of the line is discarded.


Line 9 draws the slashes on the wires (a common way to indicate that a
wire carries more than one qubit); the list of wires comes first, and the
optional argument at the end is written in math mode in the diagram.
(For \qpic's purposes this is a ``gate'', even though it does not
correspond to any computation.)  This line also contains specifications
for the width and height of the gate; these can be applied to any gate to
override the defaults, most commonly to accommodate extra text.


The next few lines incorporate another \qpic capability:\ comments.  Text after
a percent sign is written to the left of the circuit (or above a horizontal
circuit); if there is a second percent sign, that text is written to the
right of (or below) the circuit.  As we see in this example, the comments
are processed by \LaTeX.  Long comments tend to work better with
vertical circuits.


Aside from the comments, lines 10, 11, and 14 contain no new ideas; we
have several {\tt G} gates, one with a control and a width parameter.
Line 12 introduces a measurement operator; this draws a meter on each
specified wire.  If an optional argument follows the {\tt M}, the meter
is replaced by a bullet shape with the specified text inside; see
Section~\ref{sec-qpic-measure} for an example.


A \qpic wire can have one of three types:\ {\tt qwire} (quantum, or single line),
{\tt cwire} (classical, or double line), and {\tt owire} (off).
Since measurement changes a wire from quantum to classical, the {\tt M}
operator automatically changes any affected wires from {\tt qwire} to
{\tt cwire}.  As we will see in Section~\ref{sec-example-teleport},
we can also change wire types (and colors and styles) directly.


As noted in Section~\ref{sec-example-QFT}, \qpic tries to place gates
within the same slice when possible---for example, the gates on
lines 12 and 14 could be drawn next to each other.  In this diagram
the comments would then be on top of each other, which would look bad.
The solution is line 13, {\tt x TOUCH}, which is essentially a
``no-op''.  It tells \qpic to ``touch''
the {\tt x} wire at this time (i.e., during the measurement), forcing
the final gate to occur in a later slice.  {\tt TOUCH} with no
arguments would touch all wires.  See Section~\ref{sec-qpic-depth} for
some other ways to manipulate which gates occur at the same time.

\subsection{Example 4:\ Teleportation}
\label{sec-example-teleport}

In Figure~\ref{fig-teleport} we introduce a few more things \qpic can do: color, changing wire styles with a colon, and specifying ranges of slices.

\begin{figure}[h!] 
\begin{center}
\input teleport.tikz
\begin{minipage}{5in}
\lstinputlisting{teleport.qpic}
\end{minipage}
\end{center}
\caption{Teleportation from {\color{red}Alice} to {\color{blue}Bob}:\ diagram and code.}
\label{fig-teleport}
\end{figure}

The first use of the {\tt color=} syntax is when we declare wire {\tt 0}
in line 5.  This sets the (initial) color of the wire to red.
Similarly, line 13 specifies a red gate.  One can use the similar {\tt style=}
syntax to pass style commands to \TikZ; this can change the
thickness of a wire or gate.  (See Section~\ref{sec-qpic-style} for details.)

For Bob's gates on lines 17 and 18, we instead use a macro, {\tt Bob},
which we defined in line 3.  Roughly speaking, the
{\tt DEFINE} command tells \qpic to replace the next word ({\tt Bob})
with the rest of the line ({\tt color=blue}) wherever it is used.  (Note that
this expansion affects \qpic commands, but not the comments on lines 11 and 23.)
In the bottom two gates, the connectors are blue (the gate color) but drawn
with double lines (since the controls are classical).

Line 10 uses the {\tt :} syntax to change the colors of the two wires.  This
means that, as of this gate, {\tt color=red} is applied to wire 1, and
{\tt Bob} (i.e., {\tt color=blue}) is applied to wire 2.  Lines 17 and 18 use
the same syntax to change the wire type to {\tt owire}, which means those
wire are ``off'' (not drawn) starting from those points.

The remaining commands refer to slices; the first set of gates is
considered slice 0, then slice 1, and so on.  In line 11,
{\tt @ 2} specifies the most recent 2 slices (i.e., 0 and 1).  This
command does nothing by itself, but it lets us attach a comment to this
part of the circuit.  In lines 16 and 19 the two numbers following the
{\tt @} are the first and last slices in the range.  Be aware that
\qpic is happy to draw comments on top of each other; it is up to
the user to position them appropriately.

The {\tt CUT} command tells \qpic to put ``cut here'' lines before
the specified slices; in line 12, we place a single line before
slice 2.  If no arguments are given, {\tt CUT} draws dashed lines
between all pairs of slices.

\section{\qpic Commands}
\label{sec-qpic}

\lstset{basicstyle=\ttfamily\small,numbers=none}

Where Section~\ref{sec-example} serves as a tutorial, this section serves
as a reference.  We give a complete description of all commands in \qpic,
sorted by categories.  Where applicable, we give an example together with
a small diagram; in all cases, the example is a valid input that
generates the accompanying diagram.

Throughout, we use ``gate'' in the \qpic sense to mean
something that is drawn on a wire, whether or not it corresponds
to some quantum operation.

\subsection{Wires}
\subsubsection{Wire Declarations}
\label{sec-qpic-wire-name-rules}

\begin{description}
\item[{\tt name W [labels]}] Declare a wire with the given name.  If one label is
  given, it is used at the start; if two labels are given, the second is used at the
  end.  Additional labels are normally ignored, but may appear if {\tt START} and
  {\tt END} are used (see Section~\ref{sec-qpic-start-end}).  Subsequent declarations
  of the same wire simply append additional labels (if any) to the wire's list.

\begin{minipage}[b]{2in}
\lstinputlisting[firstline=2,lastline=4]{ex.W.qpic}
\end{minipage} \hfill \input ex.W.tikz
\item[{\tt ...name W}] Declare an ellipsis wire with the given name.

\begin{minipage}[b]{2in}
\lstinputlisting[firstline=2,lastline=4]{ex.ellipsis.qpic}
\end{minipage} \hfill \input ex.ellipsis.tikz

\item[{\tt names W [labels]}] The specified labels are applied to all the wires
  listed.  If a label begins and/or ends with {\tt <} or {\tt >}, it is instead
  drawn with an open or close brace before and/or after it.
  
\begin{minipage}[b]{2.7in}
\lstinputlisting[firstline=2,lastline=5]{ex.nW.qpic}
\end{minipage} \hfill \input ex.nW.tikz
\end{description}

The wire name does not appear in the diagram, but is used by \qpic to identify the wire.
One useful convention is for wire names to be made up of numbers, lowercase letters,
and underscores.  Technically, a wire name can be any non-whitespace string, except:
\begin{itemize}
\item Wire names should not contain any of the characters that \qpic handles specially:
{\tt \char92\#\$\{\}\%=@":;}.
\item  Any name of a built-in or user-defined \qpic command is a reserved word and may not be used.
\item  A wire name may not begin with {\tt -} or {\tt +}.  A wire name preceded by a
  minus sign is interpreted as a negated control; one preceded by a plus sign is
  interpreted as a target.
\item  If a wire name starts with {\tt ...} it is treated as an ellipsis.
\end{itemize}

Wires appear in the diagram in the order they are declared.  If the same wire is declared
more than once, later declarations are ignored, except that additional labels (if any)
are appended to the list of labels associated to the wire, as in the example above.

\subsubsection{Undeclared Wires}

\qpic also allows the option of undeclared wires.  This is useful for larger,
computer-generated circuits, where you don't need the extra check to catch
typos.  By default, \qpic starts in ``autowires'' mode, allowing
these undeclared wires.  As soon as a single {\tt W} declaration is seen, \qpic
switches to a ``declared'' mode where all wires must be explicitly declared.
A typical file will use one of the two modes exclusively.

\begin{itemize}
\item A valid name for an undeclared wire is either non-negative integer,
  or a string of the form
  $$
  \text{\tt lowers[\_][num[,num[,num...]]]}
  $$
  That is, a lowercase string, an optional underscore, and then a (possibly empty)
  comma-separated list of numbers.
\item Undeclared wires appear on the page after any declared wires.  Wires with integer
  labels are listed in numerical order; other undeclared wires are then sorted first by
  lowercase string, then by number of subscripts, and then lexicographically.
\item An undeclared wire with an integer name has no label.
  Otherwise, an undeclared wire is given a starting label: the
  lowercase string, followed by the numbers as subscripts.
\item Note that \qpic interprets the undeclared wire name in terms of a
standard representation.  If
  declared, {\tt a\_0} and {\tt a00} are different strings and refer to different wires,
  but if undeclared they have the same representation and refer to the same wire.
\item The command {\tt AUTOWIRES} sets \qpic to ``autowires'' mode for the rest
  of the circuit.  This lets you mix declared and undeclared wires.
\end{itemize}

\subsection{Gates}

\subsubsection{Controlled NOT and controlled $Z$}

\begin{description}
\item[{\tt target N}] Negate the target wire.

\begin{minipage}[b]{2in}
\lstinputlisting[firstline=11]{ex.N.qpic}
\end{minipage} \hfill \input ex.N.tikz

\item[{\tt target C control}] Controlled NOT.

\begin{minipage}[b]{2in}
\lstinputlisting[firstline=11]{ex.C.qpic}
\end{minipage} \hfill \input ex.C.tikz

\item[{\tt target T control1 control2}] Toffoli gate.

\begin{minipage}[b]{2in}
\lstinputlisting[firstline=11]{ex.T.qpic}
\end{minipage} \hfill \input ex.T.tikz

\item[{\tt controls}]  If a list of wires is given without a gate, it is assumed to be a (generalized) controlled $Z$ or NOT.  A target of NOT is specified
  with a {\tt +}.

\begin{minipage}[b]{2in}
\lstinputlisting[firstline=11]{ex.none.qpic}
\end{minipage} \hfill \input ex.none.tikz
\end{description}

Note the negated control in the last example.  In general, any time a wire is
used as a control, we can use a {\tt -} to change it to a negated control.

\subsubsection{General Gates}

\qpic provides two ways to do a general gate: {\tt G} for rectangles and
{\tt P} for circles (or, more precisely, ellipses).  In theory, each of these
can be used with the right attributes (Section~\ref{sec-qpic-style}) to
simulate the other, but their default behavior is different.
      
\begin{description}
\item[{\tt targets G name [controls]}] General rectangular gate.  It must have at
least one target.

\begin{minipage}[b]{2in}
\lstinputlisting[firstline=11]{ex.G.qpic}
\end{minipage} \hfill \input ex.G.tikz

\item[{\tt targets P name [controls]}] Circular gate.  It typically has one target but may have more.

\begin{minipage}[b]{2in}
\lstinputlisting[firstline=11]{ex.P.qpic}
\end{minipage} \hfill \input ex.P.tikz

\item[{\tt targets G| name [controls]} or {\tt targets |G name [controls]}] The same idea as {\tt G}, but the indicated
side of the rectangle is thicker to indicate directionality.

\begin{minipage}[b]{2in}
\lstinputlisting[firstline=11]{ex.Gbar.qpic}
\end{minipage} \hfill \input ex.Gbar.tikz

\item[{\tt targets G name targets G name ...}] One can combine more than
  one {\tt G}, {\tt P}, {\tt G|}, or {\tt |G} into a single gate.
  Each subsequent rectangle (or ellipse) is applied to all wires since the
previous one.

\begin{minipage}[b]{3in}
\lstinputlisting[firstline=11]{ex.GG.qpic}
\end{minipage} \hfill \input ex.GG.tikz
\end{description}

In this last example, note that {\tt :width=20} applies only to that
rectangle, but the other has the default width.

\subsubsection{Other predefined Gates}

\begin{description}
\item[{\tt target H [controls]} or {\tt target X [controls]}] Hadamard or $X$ gate.
It must have exactly one target.

\begin{minipage}[b]{2in}
\lstinputlisting[firstline=11]{ex.HX.qpic}
\end{minipage} \hfill \input ex.HX.tikz

\item[{\tt target Z [controls]}] $Z$ gate.  As with Hadamard and $X$, it must
  have exactly one target.  However, controlled Z can also be depicted with
  dots.

\begin{minipage}[b]{2in}
\lstinputlisting[firstline=11]{ex.Z.qpic}
\end{minipage} \hfill \input ex.Z.tikz

It is worth noting that a two-wire gate with no controls is also useful for
sorting networks.  (See Section~\ref{sec-qpic-R} for the {\tt R} command.)

\begin{minipage}[b]{2in}
\lstinputlisting[firstline=11]{ex.S.qpic}
\end{minipage} \hfill \input ex.S.tikz

\item[{\tt target1 target2 SWAP [controls]}] Swap gate.

\begin{minipage}[b]{2in}
\lstinputlisting[firstline=11]{ex.SWAP.qpic}
\end{minipage} \hfill \input ex.SWAP.tikz

\item[{\tt +}] Any time a wire would normally be a control but is preceded
  by {\tt +}, it is drawn as a target.

\begin{minipage}[b]{2in}
\lstinputlisting[firstline=11]{ex.plus.qpic}
\end{minipage} \hfill \input ex.plus.tikz

\end{description}

\subsection{Attributes}
\label{sec-qpic-style}

Attributes are of the form {\tt attribute=value}, and can affect different
properties of a wire or gate:
\begin{itemize}
\item Size: {\tt height}, {\tt width}, {\tt length}, {\tt breadth}, {\tt size}.
\item Appearance: {\tt color}, {\tt fill}, {\tt style}.
\item Other properties: {\tt type}, {\tt hyperlink}, {\tt operator}, {\tt shape}.
\end{itemize}

Attribute names may be abbreviated; e.g., {\tt oper} for {\tt operator} or
{\tt co} for {\tt color}.  At least two letters must be used.

Attributes can be used in one of three ways:
\begin{itemize}
\item Anywhere on a line, separated from other elements by spaces.  This represents a property of the gate (or wire, if it's a {\tt W} declaration).
\item After to a gate or other circuit element, separated by a colon.  This represents a property of that particular element.
\item After a wire name in a gate, separated by a colon.  This represents a
  change in the wire's properties, effective as of that gate.  If this
  part of the circuit is repeated or reversed (see
  Section~\ref{sec-qpic-R}), \qpic will try to repeat or undo the change.
\end{itemize}
Note that not all attributes can be applied in all situations.

\subsubsection{Size Attributes}

There are two different coordinate systems for specifying gate sizes:  \emph{height} v.\ \emph{width} and
\emph{length} v.\ \emph{breadth}.  \emph{Height} is always vertical on the page, and \emph{width} is horizontal; for example,
if a rectangle needs to be larger to fit multiple characters, one can increase its width.  \emph{Length} refers to the direction
of time, and \emph{breadth} refers to the perpendicular direction; for example, if a rectangle needs to be larger to convey
that it takes a long time, one can increase its length.\footnote{For horizontal circuits, length is width and breadth is height.  For vertical circuits, length is height and breadth is width.}

\begin{description}
\item[{\tt height=value}, {\tt width=value}, {\tt length=value}, {\tt breadth=value}] Change the size of a gate.\footnote{For controlled NOT or $Z$, length affects only spacing.}
Units are in points (although scaling may change this); the default value for a rectangle is 12 in both directions.
\qpic will treat the specified value as a minimum and increase it if necessary to span the indicated wires.

\begin{minipage}[b]{1.8in}
\lstinputlisting[firstline=11]{ex.size.qpic}
\end{minipage} \hfill \input ex.size.tikz 

\begin{minipage}[b]{1.8in}
\lstinputlisting[firstline=11]{ex.sizevert.qpic}
\end{minipage} \hfill \input ex.sizevert.tikz

\item[{\tt W breadth=value}] Set the breadth of a wire.  The breadth can only be changed in the initial declaration, not during the circuit.  You can instead use {\tt width} in a vertical circuit or {\tt height} in a horizontal circuit.  You cannot set the length of a wire; it is determined by the circuit.
  Wires are automatically separated by a distance of {\tt WIREPAD}, which
  defaults to 3 (Section~\ref{sec-parameters}).

\begin{minipage}[b]{2in}
\lstinputlisting[firstline=11]{ex.breadth.qpic}
\end{minipage} \hfill \input ex.breadth.tikz

\item[{\tt size=value}] Change the height and width of a gate simultaneously.
  If applied to a target or control wire this instead changes the size of
  the target or control.

\begin{minipage}[b]{2in}
\lstinputlisting[firstline=11]{ex.setsize.qpic}
\end{minipage} \hfill \input ex.setsize.tikz
  
\item [{\tt target P width=value [controls]}] The only difference between
  {\tt P} and {\tt G} is how they are affected by size changes.
  \qpic tries to keep a single-target {\tt P} a circle; changing height or width
  affects both.  To make it an ellipse, one needs to give both
  attributes explicitly.

\begin{minipage}[b]{2in}
\lstinputlisting[firstline=11]{ex.Pwidth.qpic}
\end{minipage} \hfill \input ex.Pwidth.tikz

  
\end{description}

\subsubsection{Appearance Attributes}

\begin{description}
\item[{\tt color=value}] Change the color of a gate or wire.

\begin{minipage}[b]{3in}
\lstinputlisting[firstline=11]{ex.color.qpic}
\end{minipage} \hfill \input ex.color.tikz

\item[{\tt fill=value}] Change the fill color of part of a gate.  Wires and
  controlled nots do not have fills, but targets can.

\begin{minipage}[b]{3.3in}
\lstinputlisting[firstline=11]{ex.fill.qpic}
\end{minipage} \hfill \input ex.fill.tikz

\item[{\tt style=value}] Change the style of a gate or wire.  The style parameter is passed directly to \TikZ,
  except that underscores are replaced by spaces.  Multiple styles can be separated
  by commas (and are passed to \TikZ as a group); multiple attributes are separated by
  colons.

\begin{minipage}[b]{3in}
\lstinputlisting[firstline=11]{ex.style.qpic}
\end{minipage} \hfill \input ex.style.tikz

\end{description}

The last gate above illustrates another rule:
If a wire is specified multiple times in a single gate, only the first
appearance matters to draw the gate, but the later ones can be used to attach
attributes.  This can make the code cleaner.

\begin{minipage}[b]{3in}
\lstinputlisting[firstline=11]{ex.delay.qpic}
\end{minipage} \hfill \input ex.delay.tikz

\subsubsection{Other attributes}\label{sec-attributes-other}

\begin{description}
\item[{\tt type=value}, {\tt qwire}, {\tt cwire}, {\tt owire}] There are
  three types of wires:\ quantum (single line), classical (double line),
  or off (no line).  By default, all wires are quantum.  You can set or change
  the type using the {\tt type} command (\qpic looks only at the first letter
  of value) or using the shorthand {\tt qwire}, {\tt cwire}, {\tt owire}.
  (These have the same syntax with colons as {\tt attribute=value}.)
  {\tt type} applies only to wires, but the quantum/classical status of
  wires affects how gates are drawn.

\begin{minipage}[b]{2in}
\lstinputlisting[firstline=11]{ex.qcowire.qpic}
\end{minipage} \hfill \input ex.qcowire.tikz

\item[{\tt shape=value}] You can change the shape of
  a $G$, $P$, or target.  Value is a integer, or possibly {\tt box}, {\tt circle}, {\tt <}, or {\tt >}:
  \begin{center}
    \begin{tabular}{r|l}
      $0$ & no boundary \\
      $1$ & control \\
      $-1$ & negated control \\
      {\tt circle}, $2$ & circle \\
      {\tt box}, $n$ & $n$-sided polygon with a flat bottom/right (${\tt box}=4$) (for $n \ge 3$) \\
      $-n$ & $n$-sided polygon with a vertex on the bottom/right (for $n \ge 3$) \\
      {\tt <}, {\tt >} & triangle pointing forward or backward
    \end{tabular}
  \end{center}
  \qpic will draw a regular polygon on a single wire, or stretch the polygon if it
  spans multiple wires or if a size is specified.  In horizontal mode,
the shape is drawn either with a side or a vertex on the bottom;
in vertical mode, the side or vertex is on the right of the circuit.
  If not attached to any particular element, {\tt shape} affects
  any $G$ or $P$ on the line, or (if there is no $G$ or $P$) the target.

\begin{minipage}[b]{3in}
\lstinputlisting[firstline=11]{ex.shape.qpic}
\end{minipage} \hfill \input ex.shape.tikz

\item[{\tt operator=value}] This can be used to specify the operator for
  {\tt G}, {\tt P}, {\tt M}, or a target.  For {\tt G}, {\tt P}, {\tt M}, this is
  an alternative to having the operator be the next item on the line.

\begin{minipage}[b]{3in}
\lstinputlisting[firstline=11]{ex.operator.qpic}
\end{minipage} \hfill \input ex.operator.tikz

Certain values of {\tt operator} are treated specially.  (Note the need for a
double backslash.)
\begin{center}
  \begin{tabular}{r|ll}
    {\tt 0} & none \\
    {\tt -} {\tt |} {\tt /} {\tt \char92\char92} & a single line \\
    {\tt +} {\tt x} {\tt X} & two perpendicular lines \\
    {\tt *} & four lines if {\tt shape} is $2$, $4$, or $-4$, otherwise one line to each vertex \\
    {\tt -*} & same as {\tt *}, but one line to each side. \\    
  \end{tabular}
\end{center}

\begin{minipage}[b]{3in}
\lstinputlisting[firstline=11]{ex.operator2.qpic}
\end{minipage} \hfill \input ex.operator2.tikz

Finally, if the operator is enclosed in double quotes, it is drawn by \TikZ (with (0,0) as the
center of the desired location):

\begin{center}
\begin{minipage}[b]{5.55in}
\lstinputlisting[basicstyle=\scriptsize\ttfamily,firstline=2]{ex.operatorquotes.qpic}
\end{minipage}

\input ex.operatorquotes.tikz
\end{center}

\item[{\tt hyperlink=value}] This specifies that a gate\footnote{There is a known bug: in some older versions of \TeX, only one edge of the gate is the link.}
  is a hyperlink to a target
  elsewhere in the document, possibly declared with {\tt HYPERTARGET}
  (see Section~\ref{sec-hypertarget}).

\begin{minipage}[b]{4in}
\lstinputlisting[firstline=11]{ex.hyperlink.qpic}
\end{minipage} \hfill \input ex.hyperlink.tikz

\end{description}

\subsection{Measurement and Other Wire Type Changes}
\label{sec-qpic-measure}\label{sec-qpic-start-end}
\begin{description}

\item[{\tt wires M [name]}] Measure the wires.  If the optional argument is given, \qpic draws a D-shaped ``bullet''
containing the name; if not, \qpic draws a meter.  Measurement automatically changes its targets to {\tt cwire}.

\begin{minipage}[b]{2in}
\lstinputlisting[firstline=11]{ex.M.qpic}
\end{minipage} \hfill \input ex.M.tikz

\item[{\tt MEASURESHAPE tag}] Change the default shape of the measurement gate.
  {\tt MEASURESHAPE} must be followed by {\tt D} (the default) or {\tt tag}.
  Meters are unaffected.  Only one shape may be used throughout a circuit.

\begin{minipage}[b]{2in}
\lstinputlisting[firstline=11]{ex.Mtag.qpic}
\end{minipage} \hfill \input ex.Mtag.tikz

\item[{\tt wire:cwire}] If a wire is changed to a classical wire, \qpic draws a meter in place of a control.

\begin{minipage}[b]{2in}
\lstinputlisting[firstline=11]{ex.CHANGEcwire.qpic}
\end{minipage} \hfill \input ex.CHANGEcwire.tikz

\item[{\tt target OUT value} or {\tt target IN value}] Drop a wire out or bring it back in, generally because it has a known
value.  The wire type is changed to {\tt owire} (by {\tt OUT}) or {\tt qwire} (by {\tt IN}).

\begin{minipage}[b]{2in}
\lstinputlisting[firstline=11]{ex.INOUT.qpic}
\end{minipage} \hfill \input ex.INOUT.tikz

\item[{\tt target(s) START} or {\tt target(s) END}] Start or end a wire later in the
  circuit, possibly because it is not relevant at certain times.  The labels are not
  specified by these commands, but in a {\tt W} declaration.  The wire type is changed
  to {\tt owire} (by {\tt END}) or {\tt qwire} (by {\tt START}).  If the first of these
  gates to apply to a wire is {\tt START}, then its initial start is deferred.

\begin{minipage}[b]{2in}
\lstinputlisting[firstline=11]{ex.STARTEND.qpic}
\end{minipage} \hfill \input ex.STARTEND.tikz

Certain commands (e.g., {\tt TOUCH}, {\tt LABEL}) apply by default to all wires.  More
precisely, wires that are not currently ``active'' (as determined by {\tt START} and
{\tt END}) are excluded.

\end{description}

\subsection{Managing Slices}
\label{sec-qpic-touch}
\label{sec-qpic-depth}

\qpic greedily divides a circuit into \emph{slices}.  Each gate is placed into the earliest possible slice\footnote{Except that {\tt START} is always as late as possible.}, with
the constraint that different gates applying to the same wire must occur (in order) in different slices.
Within a slice, \qpic arranges gates as efficiently as possible without overlapping on the page.
\qpic places padding to separate slices, but subslices are immediately adjacent to one another.

Most of the time, \qpic's default behavior will look fine.  The commands in this section can be used to modify this behavior.

\begin{minipage}[b]{2in}
\lstinputlisting[firstline=11]{ex.noTOUCH.qpic}
\end{minipage} \hfill \input ex.noTOUCH.tikz


\begin{description}
\item[{\tt [wires] TOUCH}] Pretend that the given wires were ``touched'' by the most recent gate (or the last time any of its targets was used, whichever is later), forcing subsequent gates
  to be in later slices.  If no wires are given, touch all wires.  Technically,
  {\tt TOUCH} draws an invisible line, which can be made visible with attributes.

\begin{minipage}[b]{2in}
\lstinputlisting[firstline=11]{ex.TOUCH.qpic}
\end{minipage} \hfill \input ex.TOUCH.tikz

\item[{\tt [wires] PHANTOM}] The same idea as {\tt TOUCH}, but with subslices rather than slices.

\begin{minipage}[b]{2in}
\lstinputlisting[firstline=11]{ex.PHANTOM.qpic}
\end{minipage} \hfill \input ex.PHANTOM.tikz

\item[{\tt [wires] BARRIER}] The same idea as {\tt TOUCH} (i.e., affecting slices),
  but with a zigzag line.

\begin{minipage}[b]{2in}
\lstinputlisting[firstline=11]{ex.BARRIER.qpic}
\end{minipage} \hfill \input ex.BARRIER.tikz

\item[{\tt LB} and {\tt LE}] Begin and end a \emph{level}: a set of gates required to be in the
same slice.  No checking is done to see whether gates in a level use the
same wires.  Attributes placed on the {\tt LB} line will be passed on to the gates.

\begin{minipage}[b]{2in}
\lstinputlisting[firstline=11]{ex.level.qpic}
\end{minipage} \hfill \input ex.level.tikz

\item[{;}] Multiple gates may be listed on one line, separated by semicolons.
  They will all be at the same depth; more precisely, \qpic will
  enclose them in {\tt LB} and {\tt LE}.
  If any command (typically the first or last) contains nothing but an
  attribute, that attribute is passed on to all gates.

\begin{minipage}[b]{2in}
\lstinputlisting[firstline=11]{ex.semicolon.qpic}
\end{minipage} \hfill \input ex.semicolon.tikz

\item[{\tt MARK name(s)}] Place a ``mark'' with the given name at the depth of the
  most recent gate in the circuit.  This has no effect by itself, but the marks can be
  used as arguments to {\tt R} (Section~\ref{sec-qpic-R}) or {\tt @}
  (Section~\ref{sec-comment}).

\item[{\tt MIXGATES value}] If {\tt MIXGATES} is set to $0$, \qpic will not mix different types of gates within the
same slice.  The default value is~1.

\begin{minipage}[b]{2in}
\lstinputlisting[firstline=11]{ex.MIXGATES.qpic}
\end{minipage} \hfill \input ex.MIXGATES.tikz
\end{description}

\subsection{Reversing and Repeating}
\label{sec-qpic-R}

One common quantum operation is to reverse or repeat part of a circuit.  In \qpic, this can be done by
selecting sets of slices.  The first slice in the circuit is $0$, then $1$, and so on.  For
the {\tt R} command, $-1$ refers to the most recent slice, $-2$ to the one before that, and so on.

\begin{description}
\item[{\tt R start end}] If $\mathtt{start} < \mathtt{end}$, repeat the specified range of
slices (those between {\tt start} and {\tt end}, including both endpoints).  We replay the slices in order.

\begin{minipage}[b]{2in}
\lstinputlisting[firstline=11]{ex.R1.qpic}
\end{minipage} \hfill \input ex.R1.tikz

\item[{\tt R [start end]}] If $\mathtt{start} \ge \mathtt{end}$, reverse the specified range of slices
(those between {\tt end} and {\tt start}, including both endpoints).  We replay the slices
in reverse order, swapping {\tt G|} with {\tt |G} and undoing wire changes.  If the endpoints are not included,
we take {\tt start} to be $-2$ and {\tt end} to be $0$ (i.e., we undo everything but the most recent slice).

\begin{minipage}[b]{2in}
\lstinputlisting[firstline=11]{ex.R2.qpic}
\end{minipage} \hfill \input ex.R2.tikz

\item[{\tt R [name name]}] We can also use names set by the {\tt MARK} command in place
  of numbers.  We will replay the slices either forward or in reverse, depending on
  whether the first or second mark is the earlier one.  If names are used, the earlier
  mark is \emph{excluded}.  (You can also specify one slice with a mark and one with\
  a number; again, if the earlier slice is a mark, it is excluded.)

\begin{minipage}[b]{2in}
\lstinputlisting[firstline=11]{ex.Rmark.qpic}
\end{minipage} \hfill \input ex.Rmark.tikz

\end{description}

\subsection{Other Circuit Elements}

These next few items are all considered ``gates'' by \qpic.  They do not
correspond to any computation, but they take up space on the page and are assigned to
slices.

\begin{description}
\item[{\tt wires / [name]}] Draw a slash across each wire; the optional argument is written (in math mode) next to the
slash.  This is sometimes used to denote the number of qubits represented by a wire.

\begin{minipage}[b]{2in}
\lstinputlisting[firstline=11]{ex.slash.qpic}
\end{minipage} \hfill \input ex.slash.tikz

\item[{\tt [targets] LABEL [labels]}] Place the given labels (in math mode) on the wires.
  If no targets are given, the command applies to all wires.
  There should be (1) one label for each wire,
  or (2) just one label (to be repeated on each wires),
  or (3) no labels (the empty label is placed on all wires).  The label {\tt ...}
  is turned into {\tt \char92 cdots}.
  In vertical circuits, labels are rotated by 90 degrees.
  The empty label is sometimes useful to pad with space, especially in conjunction
  with a {\tt length} attribute.
  
\begin{minipage}[b]{2in}
\lstinputlisting[firstline=11]{ex.label.qpic}
\end{minipage} \hfill \input ex.label.tikz

\item[{\tt [targets] = [label]}] Place a single label, centered across the specified targets.  If no targets are specified, use all wires.  If no label is specified, use $=$.

\begin{minipage}[b]{2in}
\lstinputlisting[firstline=11]{ex.equals.qpic}
\end{minipage} \hfill \input ex.equals.tikz

\item[{\tt [targets] [<|>]=[<|>] [label]}] If {\tt =} is preceded and/or followed by
  {\tt <} or {\tt >}, then a curly brace is drawn before and/or after the label, in the
  indicated direction.

\begin{minipage}[b]{2in}
\lstinputlisting[firstline=11]{ex.equals2.qpic}
\end{minipage} \hfill \input ex.equals2.tikz

\item[{\tt wires PERMUTE}] Change the order of the specified wires on the page.  The
  left-to-right order on the line will be the new top-to-bottom (or left-to-right) order.
  \qpic will draw smooth lines for each wire, with the rounded corners specified by
  the global parameter {\tt CORNERS} (Section~\ref{sec-parameters}).

\begin{minipage}[b]{2in}
\lstinputlisting[firstline=11]{ex.PERMUTE.qpic}
\end{minipage} \hfill \input ex.PERMUTE.tikz

  
\end{description}

\subsection{Comments}
\label{sec-comment}

These next few commands are not considered ``gates'' by \qpic.  They do not take up
any space
in their circuit, and their presence does not affect the locations of other elements.

\begin{description}

\item[{\tt CUT [slices]}] Place ``cut here'' (dashed) lines just before each specified slice.  If no slices
are given, place cuts in all possible places.

\begin{minipage}[b]{2in}
\lstinputlisting[firstline=11]{ex.CUT.qpic}
\end{minipage} \hfill \input ex.CUT.tikz

\item[{\tt \% [comment1] [\% comment2]}] Place comments next to the
  gate (there must be a gate on the same line).  The {\tt comment1} is
  placed above or to the left of the circuit; {\tt comment2} is placed
  below or to the right.

\begin{minipage}[b]{2in}
\lstinputlisting[firstline=11]{ex.comment.qpic}
\end{minipage} \hfill \input ex.comment.tikz

Note that there are two types of comments in \qpic:  comments in the diagram, indicated by {\tt \%} and passed on to
\LaTeX, and comments within \qpic code itself, which start with {\tt \#} and are discarded by the parser.

\item[{\tt [wires] @ [num1 [num2]]}] Specify a rectangular region.  The
  ``breadth'' dimension spans the specified wires; if none are listed, it spans all wires.
  The ``length'' dimension is determined by the arguments after the {\tt @}.  With no
  arguments, it spans the most recent slice; with one, it refers to the most recent
  {\tt num1} slices; with two, it refers to slices {\tt num1} through {\tt num2}
  (including both endpoints).  (As with {\tt R}, a name set by {\tt MARK} may be used
  for either or both endpoints, and if the earlier slice is specified by a name it
  is excluded.)

  This region can serve as a placeholder for comments, in which case the wires are
  usually not specified.

\begin{minipage}[b]{2in}
\lstinputlisting[firstline=11]{ex.at.qpic}
\end{minipage} \hfill \input ex.at.tikz

The region can also be made visible using attributes like {\tt fill}, {\tt style},
and {\tt color}.

\begin{minipage}[b]{4.5in}
\lstinputlisting[firstline=11]{ex.atfill.qpic}
\end{minipage} \hfill \input ex.atfill.tikz
\end{description}


\subsection{Macros and {\LaTeX} Code}
\label{sec-qpic-define} \label{sec-hypertarget}

The next few commands are for defining macros or for placing {\LaTeX} code before and after the \TikZ code generated
by \qpic.  Each may be repeated as often as necessary.

\begin{description}
\item[{\tt DEFINE macro text}] Creates user-defined \qpic code.  Hereafter, wherever the word {\tt macro}
appears, it will be replaced by {\tt text}.

\begin{minipage}[b]{3.2in}
\lstinputlisting[firstline=11]{ex.DEFINE.qpic}
\end{minipage} \hfill \input ex.DEFINE.tikz

\item[{\tt args DEFINE macro text}] If {\tt DEFINE} has $n$ arguments, then the
  $n$ space-delimited words preceding {\tt macro} will be used to replace those
  arguments wherever they occur (as space- or colon- or semicolon- delimited
  words) within {\tt text}.

\begin{minipage}[b]{3.2in}
\lstinputlisting[firstline=11]{ex.DEFINEargs.qpic}
\end{minipage} \hfill \input ex.DEFINEargs.tikz


\item[{\tt COLOR name r g b}] Insert a {\tt \char92 definecolor} command defining {\tt name} with the
specified rgb values (which should be between $0$ and $1$).

\item[{\tt HEADER text}] Add a line which will be interpreted by {\tt tikz2preview} to place
{\tt text} at the start of its output, before the {\tt \char92 begin\{document\}}.

\item[{\tt PREAMBLE text}] Add {\tt text} to the {\LaTeX} code just before the tikzpicture.

\item[{\tt PRETIKZ text}] Add {\tt text} at the start of the tikzpicture, before the diagram code.

\item[{\tt POSTTIKZ text}] Add {\tt text} at the end of the tikzpicture, after the diagram code.

\begin{minipage}[b]{4in}
\lstinputlisting[firstline=11]{ex_latex.qpic}
\end{minipage} \hfill \includegraphics{ex_latex}

\item[{\tt HYPERTARGET name}] Add {\tt \char92hypertarget\{name\}\{\}} at the start of
  the tikzpicture, to create a hyperref target (say, for use by {\tt hyperlink}).

\begin{minipage}[b]{3.5in}
\lstinputlisting[firstline=11]{ex.hypertarget.qpic}
\end{minipage} \hfill \input ex.hypertarget.tikz

\end{description}

One of the design principles behind \qpic is that it should solve common problems, but not all
problems.  If a user would like to include something that can't be drawn using the standard
commands, the full power of \TikZ can be accessed via {\tt PRETIKZ} and {\tt POSTTIKZ} (and via {\tt operator=''...''} in Section~\ref{sec-attributes-other}.
For this
reason, \qpic produces commented \TikZ code, to make it easy to find what coordinates should be
used to draw extra elements.

% VERTICAL, HORIZONTAL

\subsection{Global Parameters}
\label{sec-parameters}

The remaining commands set global parameters.  They can be used multiple times, but only
the final instance will have any effect.\footnote{One exception:  If {\tt VERTICAL} and {\tt HORIZONTAL}
are used throughout the circuit, this may confuse \qpic's interpretations of {\tt height=}
and {\tt width=} attributes.}

Some commands use \qpic's internal unit, which is 1 point (or $1/12$ inch).  Applying
{\tt SCALE} changes this unit size.

TO DO: Should these be command-line parameters?  If so, what's the syntax?

\begin{description}
\item[{\tt VERTICAL [deg1 [deg2]]}] Change the flow of time to vertical.  Starting labels are rotated by
{\tt deg1} degrees, ending labels by {\tt deg2} degrees; values should be between $0$ and $90$.  If {\tt deg2}
is absent, {\tt deg1} is used; if both are missing, the default is $45^\circ$.
\item[{\tt HORIZONTAL}] Change the flow of time to horizontal.  This is the default.
\item[{\tt DEPTHPAD value}] Set the amount of space between slices.  The default is~$6$.
\item[{\tt WIREPAD value}] Set the amount of space between wires.  The default is~$3$.
\item[{\tt GATESIZE value}] Set the default height and width of a gate.  The default is~$12$.
\item[{\tt COMMENTSIZE value}] Set the width of the comment region outside the circuit.  The default is~$144$.
\item[{\tt SCALE value}] Set an overall scale factor for graphical elements in the circuit (but not for text).
  The default is~$1$.
\item[{\tt MEASURESHAPE value}] Set the shape of an measurement with a label.  {\tt value}
  can be {\tt D} or {\tt tag}.  The default is {\tt D}.
\item[{\tt CORNERS value}] Set the argument to \TikZ's {\tt rounded corners} when wires
  are permuted ($0$ means no rounding).  The default is~$4$.
\item[{\tt OPACITY value}] Set the fill opacity for rectangles drawn with {\tt @}.
  The default is $0.2$.
\item[{\tt WIRES value}] Prepend {\tt value} to all wire labels within math mode (e.g., {\tt \char92 scriptstyle}).
The default is to prepend nothing.
\item[{\tt PREMATH value}] Prepend {\tt value} to all wire labels before math mode.  The default is to prepend nothing.
\item[{\tt POSTMATH value}] Append {\tt value} to all wire labels after math mode.  The default is to append nothing.
\item[{\tt BGCOLOR [value]}] Set the background color for the diagram.  The default is white; however, if {\tt BGCOLOR}
is called with no argument, it sets the background to {\tt bg}.  This color is not defined in standard {\LaTeX},
but in {\sc beamer} it is always equal to the background of the current template.

\end{description}

\section{\qpicpy and \LaTeX}
\label{sec-tools}

There are three major use-case scenarios for the \qpicpy program:
\begin{enumerate}
\item Create standalone PDF graphics from \qpic files.
\item Include \qpic diagrams as PDF graphics in a {\LaTeX} file.
\item Include \qpic diagrams as \TikZ code in a {\LaTeX} file.
\end{enumerate}

\noindent
Three files are used in the document compilation process: {\tt qpic.py},\linebreak {\tt tikz2preview}, and {\tt Makefile}.

\begin{itemize}
\item {\tt qpic.py} is a Python script that parses {\tt .qpic} files and produces {\tt .tikz} output.
\item {\tt tikz2preview} is a Python script that wraps {\tt .tikz} files to create stand\-alone {\tt .tex} files.
\item {\tt Makefile} provides the action rules to create the \TikZ code and PDF files for all \qpic files in the directory. Furthermore, the {\tt Makefile} assumes that the main document (as defined in the {\tt Makefile}) depends on all \qpic files in the directory.
\end{itemize}

\subsection{Create Standalone PDF Graphics from \qpic Files}

Copy the files {\tt qpic.py}, {\tt tikz2preview}, and {\tt Makefile} into the working directory and you are ready to go. To create a PDF file from the \qpic file {\tt diagram.qpic}, simply type:

\begin{minipage}{5in}
\begin{lstlisting}[basicstyle=\normalsize\ttfamily,numbers=none]
$ make diagram.pdf
\end{lstlisting}
\end{minipage}%$


\subsection{Include \qpic Diagrams as PDF Graphics in a {\LaTeX} File}


Copy the files {\tt qpic.py}, {\tt tikz2preview}, and {\tt Makefile} into the working directory of the {\LaTeX} document. In this example, we assume the name of the {\LaTeX} document is {\tt mainfile.tex}. Edit line 48 of {\tt Makefile} to reflect the main document name without the {\tt .tex} extension. If the main document has additional dependencies (e.g., other {\LaTeX} files, PNG graphics), add these to the end of the dependency list on line 53.

\begin{minipage}{5in}
\lstinputlisting[basicstyle=\footnotesize\ttfamily,numbers=left,numberstyle=\tiny,breaklines=true,firstline=47,lastline=54]{Makefile.example}
\end{minipage}


The package {\tt graphicx} is required to include the PDF graphics in {\LaTeX} documents.
Add the following line to {\tt mainfile.tex} before \verb|\begin{document}|.

\begin{minipage}{5in}
\begin{lstlisting}[basicstyle=\normalsize\ttfamily]
\usepackage{graphicx}
\end{lstlisting}
\end{minipage}


The \qpic PDF graphic {\tt diagram.pdf} is included in the document using the command:

\begin{minipage}{5in}
\begin{lstlisting}[basicstyle=\normalsize\ttfamily]
\includegraphics{diagram}
\end{lstlisting}
\end{minipage}

\subsection{Include \qpic Diagrams as \TikZ Code in a {\LaTeX} File}

The {\tt makefile} preparation is identical to the previous case; however the package requirement and input method are different. The package {\tt tikz} is required to compile \TikZ code in {\LaTeX} documents.
Add the following line to {\tt mainfile.tex} before \verb|\begin{document}|.

\begin{minipage}{5in}
\begin{lstlisting}[basicstyle=\normalsize\ttfamily]
\usepackage{tikz}
\end{lstlisting}
\end{minipage}


The \qpic \TikZ code {\tt diagram.tikz} is included in the document using the command:

\begin{minipage}{5in}
\begin{lstlisting}[basicstyle=\ttfamily]
\include{diagram.tikz}
\end{lstlisting}
\end{minipage}

\subsection{Comparing PDF and \TikZ Inclusion Methods}
These two approaches to including \qpic diagrams in {\LaTeX} files each have their strengths and weaknesses.  Both were used in the preparation of this document.


Advantages of PDF inclusion:
\begin{itemize}
\item Only modified graphics need to be recompiled, resulting in a faster {\LaTeX} compilation. With \TikZ inclusion, every \TikZ graphic must be reconstructed as part of the document build process.
\item Graphics can be scaled using the \verb|\includegraphics| command. This scaling is independent of the \qpic scaling and makes it easier to generate graphics of a specific size.
\end{itemize}


Advantages of \TikZ inclusion:
\begin{itemize}
\item The graphics are aware of the document settings, including font style, when they are created. Thus a slightly different graphic is created from the same \qpic file if it is in a slide environment as opposed to a document environment.
\item On systems where pdf{\LaTeX} is not available (or is not recent enough to support the \verb|preview| environment), PDF inclusion is not possible and \TikZ inclusion must be used.
\end{itemize}



\section{Tokenizing}
\label{sec-tokens}

The parsing rules for \qpic can be a bit confusing.  Both {\tt \#} and {\tt \%} are used to delimit different
types of comments.  Backslash is not technically an escape character---it is passed on to \LaTeX---but \qpic
uses it for parsing; for example, {\tt \char92\#} is not treated as initiating a comment.  For completeness,
we simply list, in order, the steps \qpic takes to parse a line.

\begin{enumerate}
\item Split the line into \emph{entities}.  Typically, an entity is a character.  When {\tt \char92} is found, it
is combined with the following character as a single entity.  Also, text within dollar signs, braces, or double quotes (which
may be nested) is combined into a single entity, even if it contains whitespace.
\item Discard the first entity equal to {\tt \#} and anything following it.
\item Split the line using the entity {\tt \%}.  The second and third portions will be used as comments;
  all subsequent parsing applies only to the first portion.
\item Group the entities into \emph{subwords} by splitting on whitespace,
  colons, and semicolons.  (A colon or semicolon is considered its own
  subword; whitespace is ignored.)  A \emph{word} is a collection of
  colon-delimited subwords.
\item If any subword is a user-defined macro (see Section~\ref{sec-qpic-define}), replace the macro with its expansion.\footnote{Except that if the
  user-defined macro is immediately preceded by {\tt DEFINE} then this is
  treated as a redefintion and the macro is not expanded.}  If the macro
  has $n$ arguments, the $n$ preceding words are removed and then used
  in place of the arguments in the macro expansion.
\item If one of the words is {\tt DEFINE}, then define the macro; the next word
  will hereafter expand to the rest of the line.  Any words preceding
  {\tt DEFINE} are arguments to the macro.
\item Split the line using the word {\tt ;}.  Each portion will be processed
  as a different command, and all these commands will be grouped inside a
  {\tt LB} and {\tt LE}.  Subsequent parsing applies separately to each command.
\item Pull out any attribute specifiers ({\tt attribute=value} for attributes
  discussed in Section~\ref{sec-qpic-style}, or {\tt qwire}, {\tt cwire},
  {\tt owire}).  Remember these so they can be passed to the appropriate
  gate or wire.  (If attributes occur in an otherwise empty semicolon-delimited
  command, they are attached to the implicit {\tt LB}.)  Complain if a subword
  following a colon is not an attribute specifier.
\item Parse the remaining words as one of the commands in
  Section~\ref{sec-qpic}.  First, check if the first word is a \qpic command.
  If not, assume the first word is a wire, and search for a word that is a
  valid \qpic gate.  If there is none, interpret the line as a (controlled) Z
  or NOT (depending on targets).
\end{enumerate} 

% Gory details of tokenizer
% If other gory details needed, then ``Gory details'' section with tokenizing subsection

\bibliography{qpic}
\bibliographystyle{plain}

\end{document}


